\documentclass{article}
\usepackage{zhspacing}
\zhspacing
\usepackage{zhulem}
\usepackage{gezhu}
\makeatletter
%\gezhu@lines=3
\begin{document}


\makeatletter
\gezhu@makespecials
\loop
\noindent\hrulefill\par
\the\hsize\par
\begin{withgezhu}
	\parskip=30pt

世祖光武皇帝讳秀,字文叔,|{礼“{祖有功而宗有德}”,光武中兴,故庙称世祖。谥法
:“能绍前业曰光,克定祸乱曰武。”伏侯古今注曰:“秀之字曰茂。伯、仲、叔、季
,兄弟之次。长兄伯升,次仲,故字文叔焉。”}南阳蔡阳人,|{南阳,郡,今邓州县
也。蔡阳,县,故城在今随州枣阳县西南。}高祖九世之孙也,出自景帝生长沙定王发
。|{长沙,郡,今潭州县也。}发生舂陵节侯买,|{舂陵,乡名,本属零陵泠道县,在
今永州唐兴县北,元帝时徙南阳,仍号舂陵,故城在今随州枣阳县东。事具宗室四王传
。}买生郁林太守外,|{郁林,郡,今贵州县。前书曰:“郡守,秦官。秩二千石。景
帝更名太守。”}外生钜鹿都尉回,|{钜鹿,郡,今邢州县也。前书曰:“都尉,本{郡
尉},秦官也。掌佐守,典武职,秩比二千石。景帝更名都尉。”}回生南顿令钦,|{南
顿,县,属汝南郡,故城在今陈州项城县西。前书曰:“令、长,皆秦官也。万户以上
为令,秩千石至六百石;不满万户为长,秩五百石至三百石。”}钦生光武。光武年九
岁而孤,养于叔父良。

身长七尺三寸,美须眉,大口,隆准,日角。|{隆,高也。许负云:“鼻头为准。”郑
玄尚书中候注云:“日角谓庭中骨起,状如日。”}性勤于稼穑,|{种曰稼,敛曰穑。}
而兄伯升好侠养士,常非笑光武事田业,比之高祖兄仲。|{仲,合阳侯喜也,能为产业
。见前书。}王莽天凤中,|{王莽始建国六年改为天凤。}乃之长安,受尚书,略通大义
。|{东观记曰:“受尚书于中大夫庐江许子威。资用乏,与同舍生韩子合钱买驴,令从
者僦,以给诸公费。”}
\end{withgezhu}
  \ifdim\hsize > 5cm
    \advance\hsize by -4pt
\repeat
\end{document}
